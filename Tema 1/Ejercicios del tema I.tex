%%
% Modificación de una plantilla de Latex para adaptarla al castellano.
%%

%%%%%%%%%%%%%%%%%%%%%
% Thin Sectioned Essay
% LaTeX Template
% Version 1.0 (3/8/13)
%
% This template has been downloaded from:
% http://www.LaTeXTemplates.com
%
% Original Author:
% Nicolas Diaz (nsdiaz@uc.cl) with extensive modifications by:
% Vel (vel@latextemplates.com)
%
% License:
% CC BY-NC-SA 3.0 (http://creativecommons.org/licenses/by-nc-sa/3.0/)
%
%%%%%%%%%%%%%%%%%%%%%

%----------------------------------------------------------------------------------------
%	PACKAGES AND OTHER DOCUMENT CONFIGURATIONS
%----------------------------------------------------------------------------------------

\documentclass[a4paper, 11pt]{article} % Font size (can be 10pt, 11pt or 12pt) and paper size (remove a4paper for US letter paper)

\usepackage[protrusion=true,expansion=true]{microtype} % Better typography
\usepackage{graphicx} % Required for including pictures
\usepackage[usenames,dvipsnames]{color} % Coloring code
\usepackage{wrapfig} % Allows in-line images
\usepackage[utf8]{inputenc}
\usepackage{enumerate}
\usepackage{enumitem}

% sudo apt-get install texlive-lang-spanish
\usepackage[spanish]{babel} % English language/hyphenation
\selectlanguage{spanish}
% Hay que pelearse con babel-spanish para el alineamiento del punto decimal
\decimalpoint
\usepackage{dcolumn}
\newcolumntype{d}[1]{D{.}{\esperiod}{#1}}
\makeatletter
\addto\shorthandsspanish{\let\esperiod\es@period@code}
\makeatother

\usepackage{longtable}
\usepackage{tabu}
\usepackage{supertabular}

\usepackage{multicol}
\newsavebox\ltmcbox

% Para algoritmos
\usepackage{algorithm}
\usepackage{algorithmic}
\usepackage{amsthm}

% Símbolos matemáticos
\usepackage{amssymb}
\usepackage{accents}
\let\oldemptyset\emptyset
\let\emptyset\varnothing

\usepackage[section]{placeins} % Para gráficas en su sección.
\usepackage[T1]{fontenc} % Required for accented characters
\newenvironment{allintypewriter}{\ttfamily}{\par}
\setlength{\parindent}{0pt}
\parskip=8pt
\linespread{1.05} % Change line spacing here, Palatino benefits from a slight increase by default

\makeatletter
\renewcommand\@biblabel[1]{\textbf{#1.}} % Change the square brackets for each bibliography item from '[1]' to '1.'
\renewcommand{\@listI}{\itemsep=0pt} % Reduce the space between items in the itemize and enumerate environments and the bibliography
\newcommand{\imagen}[2]{\begin{center} \includegraphics[width=90mm]{#1} \\#2 \end{center}}

\renewcommand{\maketitle}{ % Customize the title - do not edit title and author name here, see the TITLE block below
\begin{center} % Right align
{\LARGE\@title} % Increase the font size of the title
\end{center}
\vspace{50pt} % Some vertical space between the title and author name
\begin{flushright}
{\large\@author} % Author name
\\\@date % Date

\vspace{40pt} % Some vertical space between the author block and abstract
\end{flushright}
}

%----------------------------------------------------------------------------------------
%	TITLE
%----------------------------------------------------------------------------------------

\title{\textbf{Análisis Matemático II}\\ % Title
Relación de ejercicios del tema I} % Subtitle

\author{\textsc{Óscar Bermúdez, José Pimentel} % Author
\\{\textit{Universidad de Granada}}} % Institution

\date{\today} % Date

%----------------------------------------------------------------------------------------

\begin{document}

\maketitle % Print the title section

\section*{Sucesiones de funciones}
\begin{enumerate}
	\item Estudia la convergencia uniforme en intervalos de la forma $[0,a]$ y $[a, +\infty[$,
	donde $a > 0$, de la sucesión de funciones $\{f_n\}$ definidas para todo $x \geq 0$ por:
	$$f_n(x) = \frac{2nx^2}{1+n^2x^4}$$
	\subsection*{Soluciones:}
		\subsubsection*{Solución 1:}
		Claramente, $f_n(x) \geq 0 \quad \forall x \in \mathbb{R}^+_0$.
		
		Primero, vamos a calcular el límite puntual de $\{f_n\}$:
		
		Claramente, $f_n(0) = 0  \Rightarrow \lim\limits_{n \rightarrow +\infty} f_n(0) = 0$.
		
		Ahora, vamos a calcular $\lim\limits_{n \rightarrow +\infty} f_n(x)$ con $x > 0$,
		$$\lim\limits_{n \rightarrow +\infty} f_n(x) = \lim\limits_{n \rightarrow +\infty}
		\frac{2nx^2}{1+n^2x^4} \leq  \lim\limits_{n \rightarrow +\infty} \frac{2nx^2}{n^2x^4}
		= \lim\limits_{n \rightarrow +\infty} \frac{2}{nx^2} = 0$$
		Por tanto, $f(x) = \lim\limits_{n \rightarrow +\infty} f_n(x) =
		\left\{\begin{array}{ccc}
			0 &   si  & x = 0 \\
  			0 &  si & x > 0
  		\end{array}\right. \Rightarrow f \equiv 0$
  		
  		Pasamos ahora a ver la convergencia uniforme, para ello, debemos que comprobar que
  		$\lim\limits_{n \rightarrow +\infty} |f_n(x)-f(x)| = 0 \quad \forall x \in \mathbb{R}^+_0$.
  		
  		Como tenemos que $f \equiv 0$ y $f_n(x) \geq 0 \quad \forall x \in \mathbb{R}^+_0$,
  		el límite queda $\lim\limits_{n \rightarrow +\infty} f_n(x) = 0 \quad \forall x \in \mathbb{R}^+_0$
  		
  		Para cada $n \in \mathbb{N}$, vamos a calcular $x_n$ tal que $\displaystyle{f(x_n) =
  		\max_{x \in	\mathbb{R}^+_0}\{f(x)\}}$ ya que como tenemos que:
  		$ \left. \begin{array}{c}
  			f(x_n) \geq f_n(x) \quad \forall x \in \mathbb{R}^+_0\\
  			x_n \in \mathbb{R}^+_0
   		\end{array} \right\}
   		\Rightarrow$ Basta con comprobar $\lim\limits_{n \rightarrow +\infty} f_n(x_n) = 0$
   		
   		Si queremos buscar un máximo, como $f_n(x)$ es claramente derivable por ser composición
   		de funciones derivables, basta con calcular los puntos que anulan la derivada:
   		$$f'(x) = \frac{4nx(1+n^2x^4)-(2nx^2)4n^2x^3}{(1 + n^2x^4)^2} = 4nx \frac{1-n^2x^4}{(1+n^2x^4)^2}$$
   		$$f'(x) = 0 \Leftrightarrow x(1-n^2x^4) = 0 \Leftrightarrow
   		\left\{\begin{array}{c}
  			x = 0\\
   		  	1 + n^2x^4 = 0 \Rightarrow \displaystyle{x = \frac{1}{\sqrt{n}} = x_n}
   		\end{array}\right.$$
  		
  		De estos dos candidatos, descartamos $x = 0$ ya que $f_n(0) = 0$, $f_n(x) \geq 0$ y
  		$f_n \not \equiv 0$, y como $f'(]0,x_n[) \subset \mathbb{R}^+$ y $f'(]x_n, + \infty[)
  		\subset \mathbb{R}^-$, se tiene que $x_n$ es el máximo que buscábamos.
  		
  		Claramente, $\lim\limits_{n \rightarrow +\infty} f_n(x_n) = \lim\limits_{n \rightarrow +\infty} 
  		\frac{\displaystyle{2n\left(\frac{1}{\sqrt{n}}\right)^2}}{\displaystyle{1+n^2\left(\frac{1}{\sqrt{n}}\right)^4}}
  		= \lim\limits_{n \rightarrow +\infty} \displaystyle{\frac{2}{1+1}} = 1$
  		
  		Por tanto, los intervalos que contengan a $x_n$ no serán convergentes uniformemente.
  		
  		Vayamos ahora a los intervalos que nos piden:
  		\begin{itemize}
  			\item En el intervalo $[0,a]$, $f_n$ no converge uniformemente ya que \\
  			$\displaystyle{\lim\limits_{n \rightarrow +\infty} \max_{0 \leq x \leq a}\{f_n(x)\}
  			= \lim\limits_{n \rightarrow +\infty} f_n(x_n)} \neq 0$ esto se debe a que $\exists
  			n_0$ tal que $x_{n_0} < a \Rightarrow \forall n \geq n_0 \quad x_n < a$.
  			Por tanto, $x_n \in [0,a] \quad \forall n \geq n_0$
  			
  			\item En el intervalo $[a,+\infty[$, $f_n$ converge uniformemente ya que \\
  			$\displaystyle{\lim\limits_{n \rightarrow +\infty} \max_{x \geq a}\{f_n(x)\}
  			= \lim\limits_{n \rightarrow +\infty} f_n(a)} = 0$ se tiene que $\exists
  			n_0$ tal que $x_{n_0} < a \Rightarrow \forall n \geq n_0 \quad x_n < a$.
  			Por tanto, $x_n \notin [a,+\infty[ \quad \forall n \geq n_0$
  		\end{itemize}
  		
	\item Dado $\alpha \in \mathbb{R}$, consideremos la sucesión de funciones $\{f_n\}$, donde\\ %Salto de línea para no separar $f_n: [0,1]...$
	$f_n: [0,1] \rightarrow \mathbb{R}$ es la función definida para todo $x \in [0,1]$ por:
	$$f_n(x) = n^\alpha x(1-x^2)^n$$ 
	¿Para qué valores de $\alpha$ hay convergencia uniforme en $[0,1]$?¿Para qué valores de
	$\alpha$ hay convergencia uniforme en $[\rho,1]$, donde $\rho \in ]0,1[$?
	\subsection*{Soluciones:}
		\subsubsection*{Solución 1:}
		En primer lugar, veamos la convergencia puntual a $f \equiv 0$:
		
		Claramente, $f_n(x) \geq 0 \quad \forall x \in [0,1]$ y $f(0) = 0 = f(1)$.
		
		Ahora, veamos qué le pasa a la sucesión de funciones $\{f_n\}$ en $]0,1[:$
		$$f(x) = \lim\limits_{n \rightarrow +\infty} f_n(x) =
		\lim\limits_{n \rightarrow +\infty} n^\alpha x(1-x^2)^n =
		x \lim\limits_{n \rightarrow +\infty} n^\alpha (1-x^2)^n = 0$$
		ya que $1-x^2 \in ]0,1[ \Rightarrow |1-x^2| < 1 \Rightarrow 
		\displaystyle{\lim\limits_{n \rightarrow +\infty} (1-x^2)^n} = 0.$
			
		Pasamos ahora a ver la convergencia uniforme, para ello, debemos que comprobar que
  		$\lim\limits_{n \rightarrow +\infty} |f_n(x)-f(x)| = 0 \quad \forall x \in [0,1]$.
  		
  		Como tenemos que $f \equiv 0$ y $f_n(x) \geq 0 \quad \forall x \in [0,1]$,
  		el límite queda $\lim\limits_{n \rightarrow +\infty} f_n(x) = 0 \quad \forall x \in [0,1]$
  		
  		Para cada $n \in \mathbb{N}$, vamos a calcular $x_n$ tal que $\displaystyle{f(x_n) =
  		\max_{x \in	[0,1]}\{f(x)\}}$ ya que como tenemos que:
  		$ \left. \begin{array}{c}
  			f(x_n) \geq f_n(x) \quad \forall x \in [0,1]\\
  			x_n \in [0,1]
   		\end{array} \right\}
   		\Rightarrow$ Basta con comprobar $\lim\limits_{n \rightarrow +\infty} f_n(x_n) = 0$

   		Si queremos buscar un máximo, como $f_n(x)$ es claramente derivable por ser composición
   		de funciones derivables, basta con calcular los puntos que anulan la derivada:
   		$$f'(x) = n^\alpha (1-x^2)^n + n^\alpha \cdot x \cdot n (1-x^2)^{n-1} (-2x) =  
   		n^\alpha (1-x^2)^{n-1} \cdot [1-x^2(1+2n)] $$
   		$$f'(x) = 0 \Leftrightarrow
   		\left\{\begin{array}{c}
  			1-x^2 = 0 \Rightarrow x = 1\\
   		  	1-x^2(1+2n) = 0 \Rightarrow \displaystyle{x = \frac{1}{\sqrt{1+2n}} = x_n}
   		\end{array}\right.$$
  		
  		De estos dos candidatos, descartamos $x = 1$ ya que $f_n(1) = 0$, $f_n(x) \geq 0$ y
  		$f_n \not \equiv 0$, y como $f'(]0,x_n[) \subset \mathbb{R}^+$ y $f'(]x_n, 1[)
  		\subset \mathbb{R}^-$, se tiene que $x_n$ es el máximo que buscábamos.
  		
  		Claramente, $\lim\limits_{n \rightarrow +\infty} f_n(x_n) = \displaystyle{
  		\lim\limits_{n \rightarrow +\infty}	n^\alpha \frac{1}{\sqrt{1+2n}} 
  		\left(1-\left(\frac{1}{\sqrt{1+2n}}\right)^2\right)^n} = \displaystyle{
  		\lim\limits_{n \rightarrow +\infty}	n^\alpha \frac{1}{\sqrt{1+2n}}
  		\left(1-\frac{1}{1+2n}\right)^n} = \displaystyle{\lim\limits_{n \rightarrow +\infty}
  		n^\alpha \frac{1}{\sqrt{1+2n}} \left(\frac{2n}{1+2n}\right)^n} = \displaystyle{
  		\lim\limits_{n \rightarrow +\infty}	\frac{n^\alpha}{\sqrt{1+2n}} \cdot 
 		\exp \left(\lim\limits_{n \rightarrow +\infty} n \cdot \left(\frac{2n}{1+2n}-1\right)\right)}
 		=\\ \displaystyle{ \lim\limits_{n \rightarrow +\infty}	\frac{n^\alpha}{\sqrt{1+2n}} \cdot 
 		\exp \left(\lim\limits_{n \rightarrow +\infty} \frac{n}{1+2n}\right)} = \displaystyle{
  		\lim\limits_{n \rightarrow +\infty}	\frac{n^\alpha}{\sqrt{1+2n}} \cdot e^{\frac{1}{2}}}
 		= \\ \displaystyle{e^{\frac{1}{2}} \cdot \lim\limits_{n \rightarrow +\infty}
 		\sqrt{\frac{n^{2 \alpha}}{1+2n}}} = \displaystyle{\sqrt{\frac{e}{2}} \cdot
 		\sqrt{\lim\limits_{n \rightarrow +\infty} n^{2 \alpha - 1}}}$
 		
 		Dicho límite, se anula $\Leftrightarrow \lim\limits_{n \rightarrow +\infty} n^{2 \alpha - 1} = 0
 		\Leftrightarrow 2 \alpha - 1 < 0 \Leftrightarrow \alpha < \displaystyle{\frac{1}{2}}$
  		
  		Por tanto, $f_n(x)$ converge uniformemente en $[0,1] \Leftrightarrow \alpha < \displaystyle{\frac{1}{2}}$
  		
  		Dado $\rho \in ]0,1[$, $\exists n_0 \quad / \quad x_{n_0} < \rho \Rightarrow \forall n > n_0
  		\quad x_n < \rho \Rightarrow \\ \displaystyle{\max_{x \in [\rho,1]} \{|f_n(x) - f(x)|\} } = 
  		\displaystyle{\max_{x \in [\rho,1]} \{f_n(x)\}} = f(\rho) \rightarrow 0 \Rightarrow f_n(x)$
  		converge uniformemente en $[\rho,1] \quad \forall \alpha \in \mathbb{R}$.

	\item Para cada $n \in \mathbb{N}$, sea $f_n: \left[0, \frac{\pi}{2}\right] \rightarrow \mathbb{R}$
	la función dada por:
	$$f_n(x) = n(\cos x)^n \sin x$$
	Estudia la convergencia puntual de la sucesión de funciones $\{f_n\}$ y la convergencia
	uniforme en los intervalos $[0, a]$ y $\left[a,\frac{\pi}{2}\right]$, donde $0 < a < \frac{\pi}{2}$.
	\subsection*{Soluciones:}
		\subsubsection*{Solución 1:}
	
	\item Para cada $n \in \mathbb{N}$ sea $f_n: ]0, \pi[ \rightarrow \mathbb{R}$ la función dada por:
	$$f_n(x) = \frac{\sin^2(nx)}{n\sin x} \quad (0 < x < \pi)$$
	Estudia la convergencia puntual de la sucesión de funciones $\{f_n\}$, así como la convergencia
	uniforme en los intervalos del tipo $]0, a]$, $[a,\pi[$ y $[a,b]$, donde $0 < a < b < \pi$.
	\subsection*{Soluciones:}
		\subsubsection*{Solución 1:}
	
	\item Estudia la convergencia puntual y uniforme de la sucesión de funciones $\{f_n\}$, donde
	$f_n: \mathbb{R} \rightarrow \mathbb{R}$ está definida por:
	$$f_n(x) = \sqrt[n]{1+x^{2n}} \quad (x \in \mathbb{R})$$
	\subsection*{Soluciones:}
		\subsubsection*{Solución 1:}
	
	\item Estudia la convergencia uniforme en intervalos de la forma $]-\infty, -a]$, $[-a,a]$ y
	$[a, +\infty[$, donde $a > 0$, de la sucesión de funciones $\{f_n\}$ definidas por:
	$$f_n(x) = n \sin\left(\frac{x}{n}\right) \quad (x \in \mathbb{R})$$
	\subsection*{Soluciones:}
		\subsubsection*{Solución 1:}
	
	\item Estudia la convergencia uniformen en $\mathbb{R}^+_0$, de la sucesión de funciones $\{f_n\}$
	definidas para todo $x \in \mathbb{R}^+_0$ por:
	$$f_n(x) = \arctan\left(\frac{n+x}{1+nx}\right)$$
	\subsection*{Soluciones:}
		\subsubsection*{Solución 1:}
\end{enumerate}

\section*{Series de funciones}
\begin{enumerate}
	\item Para cada $n \in \mathbb{N}$ sea
	$$f_n(x) = \frac{x}{n^\alpha (1+nx^2)} \quad (x \geq 0)$$
	Prueba que $\sum f_n$ converge:
	\begin{enumerate}
		\item puntualmente en $\mathbb{R}^+_0$ si $\alpha > 0$.
		\item uniformemente en semirrecas cerradas que no contienen al 0.
		\item uniformemente en $\mathbb{R}^+_0$ si $\alpha > \frac{1}{2}$.
	\end{enumerate}
	\subsection*{Soluciones:}
		\subsubsection*{Solución 1:}
	
	\item Estudia la convergencia puntual y uniforme de la serie $\sum f_n$, donde $f_n: \mathbb{R}
	\rightarrow \mathbb{R}$ es la función dada por:
	$$f_n(x) = \frac{x}{1+n^2x^2} \quad (n = 0, 1, 2, \dots)$$
	Sea $F(x)= \displaystyle{\sum^{\infty}_{n = 0} f_n}$, la función suma de la serie. Calcula
	$\lim\limits_{x \rightarrow 0^-} F(x)$ y $\lim\limits_{x \rightarrow 0^+} F(x)$. \footnote{Sugerencia:\\
	Para $x > 0$ se tiene que
	$$\int^{k+1}_{k} \frac{x}{1+t^2x^2} dt \leq f_k(x) = \int^{k+1}_{k} \frac{x}{1+k^2x^2} dt \leq
	\int^{k}_{-k} \frac{x}{1+t^2x^2} dt$$}
	\subsection*{Soluciones:}
		\subsubsection*{Solución 1:}
	
	\item Estudia la convergencia puntual y uniforme de la serie $\sum f_n$, donde
	$$f_n(x)=\frac{n^{n+1}}{n!} x^n e^{-nx} \quad (x \geq 0)$$
	\subsection*{Soluciones:}
		\subsubsection*{Solución 1:}
	
	\item En cada uno de los siguientes ejercicios se especifica un conjunto $A \subset \mathbb{R}$ y,
	para cada $n \in \mathbb{N}$, se define una función $f_n: A \rightarrow \mathbb{R}$. Se pide estudiar,
	en cada caso, la convergencia puntual en $A$ de la serie de funciones $\sum f_n$, y la continuidad
	de la función suma $\displaystyle{F = \sum^{\infty}_{n = 1} f_n}$:
	\begin{enumerate}
		\item $A = \mathbb{R}$ y $f_n(x) = e^{-nx}$.
		\item $A = \mathbb{R}$ y $f_n(x) = (-1)^n \cdot \displaystyle{\frac{\sin (n^2x)}{n(\log(n+1))^2}}$.
		\item $A = \mathbb{R} \setminus \mathbb{Z}^*$ y $f_n(x) = \displaystyle{\frac{1}{n^2-x^2}}$.
		\item $A = \mathbb{R} \setminus \{-1,1\}$ y $f_n(x) = \displaystyle{\frac{x^{2n}}{1-x^{2n+1}}}$.
	\end{enumerate}
	\subsection*{Soluciones:}
		\subsubsection*{Solución 1:}
	
	\item Estudia la derivabilidad de la función de Riemann $\xi: ]1, +\infty[ \rightarrow \mathbb{R}$, definida
	para todo $x > 1$ por:
	$$\displaystyle{\xi(x) = \sum^{\infty}_{n=1} \frac{1}{n^x}}$$
	Justifica también que $\lim\limits_{x \rightarrow 1} \xi(x) = +\infty$.
	\subsection*{Soluciones:}
		\subsubsection*{Solución 1:}
\end{enumerate}

\section*{Series de potencias}
\begin{enumerate}
	\item Calcula el radio de convergencia de cada una de las series de potencias $\displaystyle{\sum_n a_n x^n}$
	y estudia el comportamiento de la serie en los extremos del intervalo de convergencia, en los siguientes casos:
	\begin{enumerate}
		\item $\displaystyle{a_n = \frac{1}{\log (n+2)}}$
		\item $a_n = (n+1)^{\log (n+1)}$
		\item $\displaystyle{a_n = e - \left(1+ \frac{1}{n}\right)^n}$
		\item $\displaystyle{a_n = \frac{1 \cdot 3 \cdot 5 \cdot \dots \cdot (2n+1)}{2 \cdot 4 \cdot 6 \cdot \dots \cdot (2n)}}$
		\item $a_n = a^{\sqrt{n}} \quad (a > 0)$
		\item $\displaystyle{a_n = 1 + \frac{1}{2} + \dots + \frac{1}{n}}$
	\end{enumerate}
	\subsection*{Soluciones:}
		\subsubsection*{Solución 1:}
	
	\item Calcula la función suma de las series de potencias $\displaystyle{\sum_{n \geq 0} (n+1) \frac{x^{3n}}{2^n}}$
	y $\displaystyle{\sum_{n \geq 1} \frac{n(x+3)^{3n}}{2^n}}$.
	\subsection*{Soluciones:}
		\subsubsection*{Solución 1:}
	
	\item Expresa la función suma de las series de potencias $\displaystyle{\sum_{n \geq 1} nx^{n-1}}$ y
	$\displaystyle{\sum_{n \geq 1} \frac{n}{n+1}x^n}$ por medio de funciones elementales y calcula el valor de
	$\displaystyle{\sum^{\infty}_{n = 1} \frac{n}{2^n(n+1)}}$.
	\subsection*{Soluciones:}
		\subsubsection*{Solución 1:}
	
	\item Calcula el radio de convergencia y la suma de las series:
	\begin{itemize}
		\item $\displaystyle{\sum_{n \geq 0} \frac{n^3+n+3}{n+1}x^n}$
		\item $\displaystyle{\sum_{n \geq 0} \frac{n^3}{n!}x^n}$
		\item $\displaystyle{\sum_{n \geq 1} \frac{1}{1 + 2 + \dots + n}x^n}$
	\end{itemize}
	\subsection*{Soluciones:}
		\subsubsection*{Solución 1:}
	
	\item Calcula la función suma de la serie de potencias $\displaystyle{\sum_{n \geq 1} \frac{1}{n(2n+1)}x^n}$ y
	deduce el valor de las sumas de las series $\displaystyle{\sum_{n \geq 1} \frac{1}{n(2n+1)}}$ y
	$\displaystyle{\sum_{n \geq 1} \frac{(-1)^n}{n(2n+1)}}$.
	\subsection*{Soluciones:}
		\subsubsection*{Solución 1:}
	
	\item Calcula la función suma de la serie de potencias $\displaystyle{\sum_{n \geq 1} \frac{x^{2n}}{n(2n-1)}}$.
	\subsection*{Soluciones:}
		\subsubsection*{Solución 1:}
	Primero vemos para qué valores de $x$ la serie converge, para ello usamos el criterio del cociente:
	
	Sea $a_n = \displaystyle{\frac{\vert x\vert^{2n}}{n(2n-1)}}$, entonces:
	$$\frac{a_{n+1}}{a_n} = \frac{|x|^{2n+2}}{(n+1)(2n+1)} : \frac{|x|^{2n}}{n(2n-1)} =
	|x|^2 \frac{n(2n-1)}{(n+1)(2n+1)} \rightarrow |x|^2$$
	
	Por tanto, solo converge si $x \in (-1,1)$, hemos averiguado ya el intervalo de convergencia y podemos
	definir la función suma $f: (-1,1) \rightarrow \mathbb{R}$ como:
	$$f(x)=\sum_{n \geq 1}\frac{x^{2n}}{n(2n-1)}$$
	
	Para calcular $f(x)$, haremos uso del resultado que nos asegura que la derivada y la integral de una
	serie de potencias es la derivada o la integral de sus términos.
	
	Por tanto, para $x \in (-1,1)$ vemos que:
	$$f'(x)=\sum_{n = 2}^\infty \frac{x^{2n-1}}{2n-1}$$
	$$f''(x)=\sum_{n=1}^\infty 2x^{2n-2}=\sum_{n=0}^\infty 2x^{2n}=\frac{2}{1-x^2}$$
	
	Como vemos, $f(0) = f'(0) = 0$ y la podemos escribir como:
	$$f' = \int_0^x f'' = \int_0^x \frac{2}{1-t^2} \quad dt = \log(1+x) - \log(1-x)$$
	
	Y por último, podemos calcular $f(x) \quad \forall x \in (-1,1)$ como:
	$$f(x) = \int_0^x f' = \int_0^x (\log(1+t) - \log(1-t)) \quad dt =$$ $$= (1+x) \log(1+x) - (1-x)\log(1-x)$$
	 
	\item Prueba que las funciones definidas por:
	\begin{itemize}
		\item $\displaystyle{g(x) = \frac{\sin x}{x}, \quad g(0) = 1}$
		\item $\displaystyle{f(x) = \frac{e^x-1}{x}, \quad f(0) = 1}$
	\end{itemize}
	son de clase $\mathit{C}^\infty$ en su intervalo natural de definición.
	\subsection*{Soluciones:}
		\subsubsection*{Solución 1:}
	
	\item Calcula el desarrollo en serie de potencias centrada en un punto $a$ de la función:
	$$\displaystyle{f(x) = \frac{2x^3-x^2+2x-7}{x^4-x^3-3x^2+x+2}}$$
	\subsection*{Soluciones:}
		\subsubsection*{Solución 1:}
	
\end{enumerate}

\end{document}
