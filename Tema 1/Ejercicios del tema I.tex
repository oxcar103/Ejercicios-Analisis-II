%%
% Modificación de una plantilla de Latex para adaptarla al castellano.
%%

%%%%%%%%%%%%%%%%%%%%%
% Thin Sectioned Essay
% LaTeX Template
% Version 1.0 (3/8/13)
%
% This template has been downloaded from:
% http://www.LaTeXTemplates.com
%
% Original Author:
% Nicolas Diaz (nsdiaz@uc.cl) with extensive modifications by:
% Vel (vel@latextemplates.com)
%
% License:
% CC BY-NC-SA 3.0 (http://creativecommons.org/licenses/by-nc-sa/3.0/)
%
%%%%%%%%%%%%%%%%%%%%%

%----------------------------------------------------------------------------------------
%	PACKAGES AND OTHER DOCUMENT CONFIGURATIONS
%----------------------------------------------------------------------------------------

\documentclass[a4paper, 11pt]{article} % Font size (can be 10pt, 11pt or 12pt) and paper size (remove a4paper for US letter paper)

\usepackage[protrusion=true,expansion=true]{microtype} % Better typography
\usepackage{graphicx} % Required for including pictures
\usepackage[usenames,dvipsnames]{color} % Coloring code
\usepackage{wrapfig} % Allows in-line images
\usepackage[utf8]{inputenc}
\usepackage{enumerate}
\usepackage{enumitem}

% sudo apt-get install texlive-lang-spanish
\usepackage[spanish]{babel} % English language/hyphenation
\selectlanguage{spanish}
% Hay que pelearse con babel-spanish para el alineamiento del punto decimal
\decimalpoint
\usepackage{dcolumn}
\newcolumntype{d}[1]{D{.}{\esperiod}{#1}}
\makeatletter
\addto\shorthandsspanish{\let\esperiod\es@period@code}
\makeatother

\usepackage{longtable}
\usepackage{tabu}
\usepackage{supertabular}

\usepackage{multicol}
\newsavebox\ltmcbox

% Para algoritmos
\usepackage{algorithm}
\usepackage{algorithmic}
\usepackage{amsthm}

% Símbolos matemáticos
\usepackage{amssymb}
\usepackage{accents}
\let\oldemptyset\emptyset
\let\emptyset\varnothing

\usepackage[section]{placeins} % Para gráficas en su sección.
\usepackage[T1]{fontenc} % Required for accented characters
\newenvironment{allintypewriter}{\ttfamily}{\par}
\setlength{\parindent}{0pt}
\parskip=8pt
\linespread{1.05} % Change line spacing here, Palatino benefits from a slight increase by default

\makeatletter
\renewcommand\@biblabel[1]{\textbf{#1.}} % Change the square brackets for each bibliography item from '[1]' to '1.'
\renewcommand{\@listI}{\itemsep=0pt} % Reduce the space between items in the itemize and enumerate environments and the bibliography
\newcommand{\imagen}[2]{\begin{center} \includegraphics[width=90mm]{#1} \\#2 \end{center}}

\renewcommand{\maketitle}{ % Customize the title - do not edit title and author name here, see the TITLE block below
\begin{center} % Right align
{\LARGE\@title} % Increase the font size of the title
\end{center}
\vspace{50pt} % Some vertical space between the title and author name
\begin{flushright}
{\large\@author} % Author name
\\\@date % Date

\vspace{40pt} % Some vertical space between the author block and abstract
\end{flushright}
}

%----------------------------------------------------------------------------------------
%	TITLE
%----------------------------------------------------------------------------------------

\title{\textbf{Análisis Matemático II}\\ % Title
Relación de ejercicios del tema I} % Subtitle

\author{\textsc{Óscar Bermúdez} % Author
\\{\textit{Universidad de Granada}}} % Institution

\date{\today} % Date

%----------------------------------------------------------------------------------------

\begin{document}

\maketitle % Print the title section

\section*{Sucesiones de funciones}
\begin{enumerate}
	\item Estudia la convergencia uniforme en intervalos de la forma $[0,a]$ y $[a, +\infty[$,
	donde $a > 0$, de la sucesión de funciones $\{f_n\}$ definidas para todo $x \geq 0$ por:
	$$f_n(x) = \frac{2nx^2}{1+n^2x^4}$$
	
	\item Dado $\alpha \in \mathbb{R}$, consideremos la sucesión de funciones $\{f_n\}$, donde\\ %Salto de línea para no separar $f_n: [0,1]...$
	$f_n: [0,1] \rightarrow \mathbb{R}$ es la función definida para todo $x \in [0,1]$ por:
	$$f_n(x) = n^\alpha x(1-x^2)^n$$
	¿Para qué valores de $\alpha$ hay convergencia uniforme en $[0,1]$?¿Para qué valores de
	$\alpha$ hay convergencia uniforme en $[\rho,1]$, donde $\rho \in ]0,1[$?
	
	\item Para cada $n \in \mathbb{N}$, sea $f_n: \left[0, \frac{\pi}{2}\right] \rightarrow \mathbb{R}$
	la función dada por:
	$$f_n(x) = n(\cos x)^n \sen x$$
	Estudia la convergencia puntual de la sucesión de funciones $\{f_n\}$ y la convergencia
	uniforme en los intervalos $[0, a]$ y $\left[a,\frac{\pi}{2}\right]$, donde $0 < a < \frac{\pi}{2}$.
	
	\item Para cada $n \in \mathbb{N}$ sea $f_n: ]0, \pi] \rightarrow \mathbb{R}$ la función dada por:
	$$f_n(x) = \frac{\sen^2(nx)}{n\sen x} \quad (0 < x < \pi)$$
	Estudia la convergencia puntual de la sucesión de funciones $\{f_n\}$, así como la convergencia
	uniforme en los intervalos del tipo $[0, a]$, $[a,\pi]$ y $[a,b]$, donde $0 < a < b < \pi$.
	
	\item Estudia la convergencia puntual y uniforme de la sucesión de funciones $\{f_n\}$, donde
	$f_n: \mathbb{R} \rightarrow \mathbb{R}$ está definida por:
	$$f_n(x) = \sqrt[n]{1+x^{2n}} \quad (x \in \mathbb{R})$$
	
	\item Estudia la convergencia uniforme en intervalos de la forma $]-\infty, a]$, $[-a,a]$ y
	$[a, +\infty[$, donde $a > 0$, de la sucesión de funciones $\{f_n\}$ definidas por:
	$$f_n(x) = n\sen\left(\frac{x}{n}\right) \quad (x \in \mathbb{R})$$
	
	\item Estudia la convergencia uniformen en $\mathbb{R}^+_0$, de la sucesión de funciones $\{f_n\}$
	definidas para todo $x \in \mathbb{R}^+_0$ por:
	$$f_n(x) = \arctan\left(\frac{n+x}{1+nx}\right)$$
\end{enumerate}

\section*{Series de funciones}
\begin{enumerate}
	\item Para cada $n \in \mathbb{N}$ sea
	$$f_n(x) = \frac{x}{n^\alpha (1+nx^2)} \quad (x \geq 0)$$
	Prueba que $\sum f_n$ converge:
	\begin{enumerate}
		\item puntualmente en $\mathbb{R}^+_0$ si $\alpha > 0$.
		\item uniformemente en semirrecas cerradas que no contienen al 0.
		\item uniformemente en $\mathbb{R}^+_0$ si $\alpha > \frac{1}{2}$.
	\end{enumerate}
	
	\item Estudia la convergencia puntual y uniforme de la serie $\sum f_n$, donde $f_n: \mathbb{R}
	\rightarrow \mathbb{R}$ es la función dada por:
	$$f_n(x) = \frac{x}{1+n^2x^2} \quad (n = 0, 1, 2, \dots)$$
	Sea $F(x)= \displaystyle{\sum^{\infty}_{n = 0} f_n}$, la función suma de la serie. Calcula
	$\lim\limits_{x \rightarrow 0^-} F(x)$ y $\lim\limits_{x \rightarrow 0^+} F(x)$. \footnote{Sugerencia:\\
	Para $x > 0$ se tiene que
	$$\int^{k+1}_{k} \frac{x}{1+t^2x^2} dt \leq f_k(x) = \int^{k+1}_{k} \frac{x}{1+k^2x^2} dt \leq
	\int^{k}_{-k} \frac{x}{1+t^2x^2} dt$$}
	
	\item Estudia la convergencia puntual y uniforme de la serie $\sum f_n$, donde
	$$f_n(x)=\frac{n^{n+1}}{n!} x^n e^{-nx} \quad (x \geq 0)$$
	
	\item En cada uno de los siguientes ejercicios se especifica un conjunto $A \subset \mathbb{R}$ y,
	para cada $n \in \mathbb{N}$, se define una función $f_n: A \rightarrow \mathbb{R}$. Se pide estudiar,
	en cada caso, la convergencia puntual en $A$ de la serie de funciones $\sum f_n$, y la continuidad
	de la función suma $\displaystyle{F = \sum^{\infty}_{n = 1} f_n}$:
	\begin{enumerate}
		\item $A = \mathbb{R}$ y $f_n(x) = e^{-nx}$.
		\item $A = \mathbb{R}$ y $f_n(x) = (-1)^n \cdot \displaystyle{\frac{\sen (n^2x)}{n(\log(n+1))^2}}$.
		\item $A = \mathbb{R} \setminus \mathbb{Z}^*$ y $f_n(x) = \displaystyle{\frac{1}{n^2-x^2}}$.
		\item $A = \mathbb{R} \setminus \{-1,1\}$ y $f_n(x) = \displaystyle{\frac{x^{2n}}{1-x^{2n+1}}}$.
	\end{enumerate}
	
	\item Estudia la derivabilidad de la función de Riemann $\xi: ]1, +\infty[ \rightarrow \mathbb{R}$, definida
	para todo $x > 1$ por:
	$$\displaystyle{\xi(x) = \sum^{\infty}_{n=1} \frac{1}{n^x}}$$
	Justifica también que $\lim\limits_{x \rightarrow 1} \xi(x) = +\infty$.
\end{enumerate}

\section*{Series de potencias}
\begin{enumerate}
	\item Calcula el radio de convergencia de cada una de las series de potencias $\displaystyle{\sum_n a_n x^n}$
	y estudia el comportamiento de la serie en los extremos del intervalo de convergencia, en los siguientes casos:
	\begin{enumerate}
		\item $\displaystyle{a_n = \frac{1}{\log (n+2)}}$
		\item $a_n = (n+1)^{\log (n+1)}$
		\item $\displaystyle{a_n = e - \left(1+ \frac{1}{n}\right)^n}$
		\item $\displaystyle{a_n = \frac{1 \cdot 3 \cdot 5 \cdot \dots \cdot (2n+1)}{2 \cdot 4 \cdot 6 \cdot \dots \cdot (2n)}}$
		\item $a_n = a^{\sqrt{n}} \quad (a > 0)$
		\item $\displaystyle{a_n = 1 + \frac{1}{2} + \dots + \frac{1}{n}}$
	\end{enumerate}
	
	\item Calcula la función suma de las series de potencias $\displaystyle{\sum_{n \geq 0} (n+1) \frac{x^{3n}}{2^n}}$
	y $\displaystyle{\sum_{n \geq 1} \frac{n(x+3)^{3n}}{2^n}}$.
	
	\item Expresa la función suma de las series de potencias $\displaystyle{\sum_{n \geq 1} nx^{n-1}}$ y
	$\displaystyle{\sum_{n \geq 1} \frac{n}{n+1}x^n}$ por medio de funciones elementales y calcula el valor de
	$\displaystyle{\sum^{\infty}_{n = 1} \frac{n}{2^n(n+1)}}$.
	
	\item Calcula el radio de convergencia y la suma de las series:
	\begin{itemize}
		\item $\displaystyle{\sum_{n \geq 0} \frac{n^3+n+3}{n+1}x^n}$
		\item $\displaystyle{\sum_{n \geq 0} \frac{n^3}{n!}x^n}$
		\item $\displaystyle{\sum_{n \geq 1} \frac{1}{1 + 2 + \dots + n}x^n}$
	\end{itemize}
	
	\item Calcula la función suma de la serie de potencias $\displaystyle{\sum_{n \geq 1} \frac{1}{n(2n+1)}x^n}$ y
	deduce el valor de las sumas de las series $\displaystyle{\sum_{n \geq 1} \frac{1}{n(2n+1)}}$ y
	$\displaystyle{\sum_{n \geq 1} \frac{(-1)^n}{n(2n+1)}}$.
	
	\item Calcula la función suma de la serie de potencias $\displaystyle{\sum_{n \geq 1} \frac{x^{2n}}{n(2n-1)}}$.
	
	\item Prueba que las funciones definidas por:
	\begin{itemize}
		\item $\displaystyle{g(x) = \frac{\sen x}{x}, \quad g(0) = 1}$
		\item $\displaystyle{f(x) = \frac{e^x-1}{x}, \quad f(0) = 1}$
	\end{itemize}
	son de clase $\mathit{C}^\infty$ en su intervalo natural de definición.
	
	\item Calcula el desarrollo en serie de potencias centrada en un punto $a$ de la función:
	$$\displaystyle{f(x) = \frac{2x^3-x^2+2x-7}{x^4-x^3-3x^2+x+2}}$$
	
\end{enumerate}

\end{document}
